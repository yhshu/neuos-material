\section{实验介绍}

本实验包含以下内容。

\begin{itemize}
	\item 实验一\ 初识实验环境
	\item 实验二\ 内核引导启动
	\item 实验三\ VGA 与串行端口
	\item 实验四\ 内存管理与分页
	\item 实验五\ 缺页异常
	\item 实验六\ 中断与系统调用
	\item 实验七\ 进程控制
	\item 实验八\ 文件系统
\end{itemize}

\subsection{NEUOS 简介}

本实验代码由 NEUOS 与 Linux v0.11 组成。 NEUOS 是一个基于 Linux v0.11 内核的教学用操作系统。通过学习早期 Linux 内核,排除当前内核中复杂而庞大的实现细节,我们能较为完整地了解内核实现原理。

由于 Linux 下的 C 语言开发通常使用 AT\&T 内联汇编,NEUOS 中的汇编代码为 AT\&T 格式。本实验需要一些简单的汇编语言知识。

\subsection{源码}

运行 NEUOS 需要在 Linux 环境下。我们提供了一个虚拟机镜像 “NEUOS Lab Environment” 用作实验环境,详见实验一关于实验环境的描述。实验环境的 Documents 目录中内置了下列三份源码,本实验在实验材料 neuos-material 中进行。

\begin{itemize}
	\item NEUOS :https://github.com/VOID001/neu-os
	\item Linux 0.11 :https://github.com/lzw429/Linux-0.11
	\item 实验材料:https://github.com/lzw429/neuos-material
\end{itemize}

\subsection{运行(README)}

NEUOS 源码使用 make 构建。打开终端,使用 cd 切换到 neu-os 目录,键入

\begin{lstlisting}
    make
\end{lstlisting}

make 将生成软盘镜像作为启动盘,这是启动 NEUOS 所必需的。

生成镜像并使用有调试功能的 Bochs 运行 NEUOS,在命令行键入:

\begin{lstlisting}
	make run_bochs
\end{lstlisting}

或者

\begin{lstlisting}
	make
	bochs -q
\end{lstlisting}

生成镜像并使用 QEMU 运行 NEUOS,在命令行键入:

\begin{lstlisting}
	make run
\end{lstlisting}

每次修改代码后,命令行键入:

\begin{lstlisting}
	make clean
	make
\end{lstlisting}

即可重新构建。

关于 QEMU 与 Bochs 的运行参数,请查看 makefile.

\subsection{预备}

完成实验所需的必备知识包括对计算机组成原理、数据结构与算法的掌握,如果了解AT\&T 汇编语言或 GNU 工具链,完成本实验会相对轻松。

\subsection{学习建议}

本实验文档中的“实验资料”,并非只为有限的实验内容而编写。实验资料对读者理解 Linux 的早期内核大有裨益。在开始每一项实验前,我们建议首先浏览写在“实验内容”后的“实验资料”。

在内核引导启动部分,建议着重关注操作系统如何与 BIOS 协调将计算机启动;在进程调度与文件系统部分,建议着重关注数据结构与算法的设计。


\subsection{参考资料}

\begin{itemize}
	\item 《Linux 内核完全注释》赵炯(对于理解 Linux 0.11 十分必要)
	\item 《Linux 内核设计与实现》Robert Love
	\item 《深入理解 Linux 内核》Daniel Bovet, Marco Cesati
	\item 英特尔® 64 位和 IA-32 架构开发人员手册 \footnote{\url{https://software.intel.com/en-us/articles/intel-sdm}} 
\end{itemize}

\subsection{实验反馈}        

请将您在本实验中遇到的疑惑或任何意见与建议反馈给我们\footnote{\url{https://github.com/lzw429/neuos-guide/issues}}。


