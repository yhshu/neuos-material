\section{实验六\ 中断与系统调用}

\subsection{实验目的}

\begin{enumerate}
    \item 加深对设备管理基本原理的认识,了解键盘中断、扫描码等概念。
    \item 掌握 Linux 使用系统调用的基本原理。
\end{enumerate}

本实验基于 Linux 0.11 源码。

\subsection{中断处理}

中断处理主要涉及两个代码文件:asm.s 和 traps.c 文件。

中断信号通常可以分为两类:\textbf{硬件中断}和\textbf{软件中断}(异常)。每个中断由 0-255 之间的一个数字来表示。软件中断是由 CPU 执行指令时探测到异常情况而引起的,通常可分为\textbf{故障}(fault)和\textbf{陷阱}(trap)两类。

在进程将控制权交给中断处理程序之前,CPU 会首先将至少 12 字节的信息压入中断处理程序的堆栈中。

由于有些异常引起中断时,CPU 内部会产生一个出错代码压入堆栈(异常中断 int 8 和 int 10-14),而其他中断并不带有这个出错代码(例如除零出错和边界检查出错)。因此对中断的处理,需要根据是否携带出错码分别处理,但处理流程是相同的。

\begin{enumerate}
    \item 所有寄存器入栈。
    \item 出错代码入栈。无出错代码时,使用 0。
    \item 中断返回地址入栈。
    \item 所有段寄存器置为内核代码段的选择符值。
    \item 调用相关 C 处理函数。
    \item 弹出入栈的出错码和后来入栈的中断返回地址。
    \item 弹出所有入栈寄存器。
    \item 中断返回。
\end{enumerate}

其中,调用的 C 函数在 kernel/traps.c 中实现。压入堆栈的出错代码和中断返回地址是用作 C 函数的参数。

\subsection{系统调用}

Linux 应用程序调用内核的功能是通过中断调用 int 0x80 进行的,寄存器 eax 中放调用号。因此该中断调用被称为系统调用。实现系统调用的文件包括 system\_call.s、fork.c、signal.c、sys.c 和 exit.c 文件,涉及时钟中断、出错停机、进程调度等系统调用。

通常名称以 'sys\_' 开头的系统调用函数都是相应系统调用需要调用的处理函数,以汇编语言实现或 C 语言实现。而名称以 'do\_' 开头的函数,可能是系统调用处理过程中通用的函数,也可能是某个系统调用专用的。

\begin{mdframed}[hidealllines=true,backgroundcolor=gray!20]
\textbf{练习 } exp6/kernel/signal.c 文件的 do\_signal 函数是系统调用中断处理程序中的信号处理程序,它将信号的处理句柄插入到用户程序堆栈中,并在系统调用结束返回后立刻执行信号句柄程序,然后继续执行用户程序。

程序将用户调用系统调用的代码指针 eip 指向信号处理句柄时,使用的语句是 "*(\&eip) = sa\_handler;"。

这条语句难道不等价于 "eip = sa\_handler;"?请尝试解释原因,注意函数内的变量声明。
\end{mdframed}

\subsection{键盘驱动}

exp6/kernel/chr\_drv/kb.S 是一个键盘驱动程序,主要包括键盘中断处理程序。kb.S 的 key\_table 标号后的代码是一张子程序地址跳转表。当取得扫描码后就根据此表调用相应的扫描码处理子程序。

例如按下 F1-F12 按键后,kb.S 的 func 标号后的汇编指令将被执行,如下所示。

\begin{lstlisting}
    pushl %eax
    pushl %ecx
    pushl %edx
    call_show_stat
    pop %edx
    pop %ecx
    pop %eax
    ...
\end{lstlisting}

可知,这段汇编调用了显示各任务状态的函数(exp7/kernel/sched.c)。

\begin{mdframed}[hidealllines=true,backgroundcolor=gray!20]
\textbf{练习 }参照 Linux v0.11 中已有的系统调用,尝试添加一个系统调用 sys\_ver。结合键盘中断,使用户在按下 F12 时,系统屏幕打印出“NEUOS exp6”,或是你编写的具有其他功能的函数。
\end{mdframed}

\subsection{fork 系统调用}

Linux 所有进程都是进程 0 的子进程。fork() 系统调用用于创建子进程。exp6/kernel/fork.c 是 exp6/kernel/system\_call.s 的辅助处理函数集,给出了 sys\_fork() 系统调用使用的两个 C 语言函数 find\_empty\_process() 和 copy\_process()。 包括进程内存区域验证与内存分配函数 verify\_area()。

\begin{mdframed}[hidealllines=true,backgroundcolor=gray!20]
\textbf{练习 }在 exp6/kernel/fork.c 中,copy\_process 的参数有 17 个,其中一个参数并没有在函数体中被使用,它对应了\textbf{堆栈}中的什么内容?请简要说明原因。

你需要熟悉函数调用时堆栈是如何被使用的,该函数由 system\_call.s 中的 system\_call 标号后的汇编指令调用。
\end{mdframed}



